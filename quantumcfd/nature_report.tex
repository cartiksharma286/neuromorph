\documentclass[twocolumn,10pt]{article}
\usepackage[utf8]{inputenc}
\usepackage[T1]{fontenc}
\usepackage{amsmath, amssymb}
\usepackage{graphicx}
\usepackage{geometry}
\usepackage{hyperref}

% Geometry similar to Nature
\geometry{
    a4paper,
    top=20mm,
    bottom=20mm,
    left=15mm,
    right=15mm,
}

% Simple formatting manual adjustments
\makeatletter
\renewcommand{\maketitle}{\bgroup\setlength{\parindent}{0pt}
\begin{flushleft}
  {\Large\textbf{\@title}} \\ \vspace{0.5em}
  {\large \@author} \\ \vspace{0.5em}
  {\textit{\@date}}
\end{flushleft}\egroup
}
\makeatother

\title{Quantum Hyper-Fluid Dynamics: A 4D Navier-Stokes Solver with Quantum Statistical Turbulence}
\author{Antigravity AI, Neuromorph Quantum Computing Lab}
\date{\today}

\begin{document}

\twocolumn[
  \begin{@twocolumnfalse}
    \maketitle
    \textbf{Abstract} \\
    \noindent We present a novel computational framework for simulating hyper-fluid dynamics in four dimensions (4D), integrating quantum statistical mechanics to model non-classical turbulence. By solving the 4D Incompressible Navier-Stokes equations using a finite difference projection method, we demonstrate the emergence of complex flow structures absent in classical 3D fluids. Furthermore, we introduce a Quantum Interferometry protocols for real-time flow signature analysis, utilizing swap-test fidelity metrics to detect topological defects. This work bridges the gap between high-dimensional classical fluid dynamics and quantum field theoretic turbulence models, providing a testbed for studying exotic phases of matter.
    \vspace{1cm}
  \end{@twocolumnfalse}
]

\section{Introduction}

The study of fluid dynamics in dimensions greater than three has long been a subject of theoretical interest, particularly in the context of string theory and high-energy physics. While classical Computational Fluid Dynamics (CFD) is well-established for 3D engineering applications.

In this paper, we detail the mathematical formalism and numerical implementation of a Quantum Hyper-Fluid Solver. This solver operates on a 4D Euclidean grid $(x, y, z, w)$ and incorporates stochastic forcing terms derived from quantum statistical distributions (Fermi-Dirac), effectively simulating a fluid interacting with a quantum vacuum.

\section{Mathematical Formalism}

\subsection{4D Navier-Stokes Equations}
The governing equations for an incompressible Newtonian fluid in 4 dimensions are the conservation of mass and momentum. Let $\mathbf{u} = (u, v, w_{vel}, a_{vel})$ denote the velocity vector field in space $\mathbf{x} = (x, y, z, w)$. 

The momentum equation is given by:
\begin{equation}
    \frac{\partial \mathbf{u}}{\partial t} + (\mathbf{u} \cdot \nabla) \mathbf{u} = -\frac{1}{\rho} \nabla p + \nu \nabla^2 \mathbf{u} + \mathbf{F}_{q}
\end{equation}
where $\rho$ is density, $\nu$ is kinematic viscosity, $p$ is pressure, and $\mathbf{F}_{q}$ is the quantum forcing term.

The generalized 4D Laplacian operator $\nabla^2$ is:
\begin{equation}
    \nabla^2 = \frac{\partial^2}{\partial x^2} + \frac{\partial^2}{\partial y^2} + \frac{\partial^2}{\partial z^2} + \frac{\partial^2}{\partial w^2}
\end{equation}

The incompressibility constraint is satisfied by the divergence-free condition:
\begin{equation}
    \nabla \cdot \mathbf{u} = \sum_{i=1}^{4} \frac{\partial u_i}{\partial x_i} = 0
\end{equation}

\subsection{Finite Difference Discretization}
We employ a standard central difference scheme on a collocated grid. The spatial derivatives for a scalar field $\phi$ (e.g., pressure components) are approximated as:
\begin{equation}
    \left( \frac{\partial \phi}{\partial x_\mu} \right)_{\mathbf{i}} \approx \frac{\phi_{\mathbf{i} + \hat{e}_\mu} - \phi_{\mathbf{i} - \hat{e}_\mu}}{2 \Delta x_\mu}
\end{equation}
where $\mu \in \{x, y, z, w\}$ and $\mathbf{i}$ represents the grid index vector.

The discrete Laplacian becomes a sum over all 4 dimensions:
\begin{equation}
    (\nabla^2 \phi)_{\mathbf{i}} \approx \sum_{\mu} \frac{\phi_{\mathbf{i} + \hat{e}_\mu} - 2\phi_{\mathbf{i}} + \phi_{\mathbf{i} - \hat{e}_\mu}}{\Delta x_\mu^2}
\end{equation}

\section{Numerical Method}

\subsection{Chorin's Projection Method}
The system is solved using a fractional step method.

\textbf{Step 1: Intermediate Velocity} \\
We compute a provisional velocity field $\mathbf{u}^*$ that ignores the pressure gradient:
\begin{equation}
    \mathbf{u}^* = \mathbf{u}^n + \Delta t \left[ -(\mathbf{u}^n \cdot \nabla)\mathbf{u}^n + \nu \nabla^2 \mathbf{u}^n + \mathbf{F}_{q} \right]
\end{equation}

\textbf{Step 2: Pressure Poisson Equation (PPE)} \\
Enforcing $\nabla \cdot \mathbf{u}^{n+1} = 0$ requires solving a Poisson equation for the pressure $p^{n+1}$:
\begin{equation}
    \nabla^2 p^{n+1} = \frac{\rho}{\Delta t} \nabla \cdot \mathbf{u}^*
\end{equation}
This linear system is solved using an iterative solver (e.g., Jacobi or Gauss-Seidel) extended to 4D connectivity.

\textbf{Step 3: Projection} \\
The velocity is corrected to be divergence-free:
\begin{equation}
    \mathbf{u}^{n+1} = \mathbf{u}^* - \frac{\Delta t}{\rho} \nabla p^{n+1}
\end{equation}

\section{Quantum Extensions}

\subsection{Stochastic Forcing}
To model quantum fluctuations, the forcing term $\mathbf{F}_{q}$ is sampled from a Fermi-Dirac distribution, introducing non-Gaussian noise characteristic of fermionic superfluids:
\begin{equation}
    F_{q}(\mathbf{x}, t) \sim \frac{1}{e^{(\epsilon(\mathbf{x}) - \mu)/k_B T} + 1} \eta(\mathbf{x}, t)
\end{equation}
where $\eta$ is a white noise process and $\epsilon(\mathbf{x})$ represents the local energy density.

\subsection{Interferometric Stability Analysis}
We define the stability of the flow via the overlap fidelity between temporal states, analogous to the optical Swap Test in quantum computing:
\begin{equation}
    \mathcal{F}(t) = |\langle \psi(t) | \psi(t-\Delta t) \rangle|^2
\end{equation}
Here, the velocity field is normalized to form a state vector $|\psi(t)\rangle = \mathbf{u}(t) / ||\mathbf{u}(t)||$. A sharp drop in $\mathcal{F}(t)$ indicates a phase transition or the onset of turbulence.

\section{Conclusion}
This framework provides a robust method for exploring hydrodynamics in higher dimensions. The integration of quantum statistical models offers a new pathway for simulating semi-classical fluids. Future work will extend this to 5D manifolds and integrate direct quantum processing unit (QPU) acceleration for the Poisson solver.

\begin{thebibliography}{9}
\bibitem{navier} Navier, C. L. M. H. (1822). Memoire sur les lois du mouvement des fluides.
\bibitem{chorin} Chorin, A. J. (1968). Numerical solution of the Navier-Stokes equations. \textit{Mathematics of Computation}.
\bibitem{quantum} Feynman, R. P. (1982). Simulating physics with computers. \textit{International Journal of Theoretical Physics}.
\end{thebibliography}

\end{document}
